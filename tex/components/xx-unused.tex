% CIE photometic curve [HDRI]
% Colorimetria
% zarovnanie vzorec
% farebne modely? sRGB, cie xyz
% HDR monitors?

%%%%%%% \subsection*{Čo ovplyvňuje dynamický rozsah scény?}

% \begin{itemize}
%     \item Na obrázku \ref{fig:scene1} vidíme scénu krajiny, zachytenú za jasného počasia s citlivosťou ISO 100
%     časom expozície 1/125 a clonou f/5.6. Ak sa na scéne nenachádza nič jasné, napríklad obloha alebo
%     jej odraz, senzor fotoaparátu zachytí celý dynamický rozsah jasu.
%     Pri vhodnej expozícii teda nedôjde k obmedzeniu záznamu svetlých ani tmavých miest.
%     Hovoríme, že scéna má malý dynamický rozsah.
%     \item Scéna \ref{fig:scene2} zachytáva krajinu s oblohou so stredným dynamickým rozsahom. Jasné slnečné
%     svetlo vytvára spolu s tieňmi väčšinu kontrastu a taktiež rozšírilo horný koniec stupnice 
%     odrazivosti bielych mrakov.
%     \item Pridajme do scény \ref{fig:scene3} Slnko priamo nad obzorom, ktoré so svojou svietivosťou
%     $10^{8}$ až $10^{9}$ $cd/m^{2}$ \cite{ZakladyHDR} nastaví hornú hranicu dynamického rozsahu poriadne
%     vysoko a prehlbí tiene proti fotoaparátu. Vysoký dynamický rozsah má taktiež scéna zachytená 
%     v miestnosti s výhľadom von.
%     \item Ak scéna \ref{fig:scene4} zachytáva aj slnečný kotúč s lesklým povrchom odrážajúcim
%     jasné slnečné svetlo, zvýši sa rozsah scény na veľmi vysoké hodnoty, teda viac ako 20 EV.
% \end{itemize}

% \begin{figure}[h!]
%     \centering
%     \begin{subfigure}{0.2\textwidth}
%         \includegraphics[width=\textwidth]{figures/theory/sceneDR1}
%         \caption{malý dynamický rozsah}
%         \label{fig:scene1}
%     \end{subfigure}
%     ~
%     \begin{subfigure}{0.2\textwidth}
%         \includegraphics[width=\textwidth]{figures/theory/sceneDR2}
%         \caption{stredný dynamický rozsah}
%         \label{fig:scene2}
%     \end{subfigure}
%     ~
%     \begin{subfigure}{0.18\textwidth}
%         \includegraphics[width=\textwidth]{figures/theory/sceneDR3}
%         \caption{vysoký dynamický rozsah}
%         \label{fig:scene3}
%     \end{subfigure}
%     ~
%     \begin{subfigure}{0.2\textwidth}
%         \includegraphics[width=\textwidth]{figures/theory/sceneDR4}
%         \caption{veľmi vysoký dynamický rozsah}
%         \label{fig:scene4}
%     \end{subfigure}
%     \caption{Dynamický rozsah scény (zdroj: \cite{ZakladyHDR})}
%     \label{fig:scene_dynamic_range}
% \end{figure}

%%%%%% TMO

% Vývoj filmovej fotografie vytvorilo problémy so zachtením enormného dynamického rozsahu 
% jasu reálneho sveta na chemicky obmedzený negatív snímacieho zariadenia. Vývojári v oblasti filmu sa 
% pokúsili vyriešiť tento problém navrhnutím filmových pásov a systémov vývoja tlače, ktoré poskytli požadovanú 
% tónovú krivku v tvare S s mierne zvýšeným kontrastom (okolo 15\%) v strednom rozsahu a postupne skomprimované 
% osvetlenia a tiene.

% Nástup digitálnej fotografie vytvoril nádej na lepšie riešenie tohto problému. Jeden z prvých algoritmov,
% ktorý využívali Land a McCann v roku 1971, bol Retinex, inšpirovaný teóriami vnímania jasu človekom. Táto metóda
% je inšpirovaná biologickými mechanizmami adaptácie oka, keď sú problémové svetelné podmienky. Algoritmy mapovacie
% gamut (dosiahnuteľná oblasť farieb v určitom farebnom priestore) boli tiež rozsiahle študované v kontexte farebnej
% tlače. Výpočtové modely ako CIECAM02 alebo iCAM sa použili na predpovedanie farebného vzhľadu. Napriek tomu, ak by 
% algoritmy nemohli dostatočne mapovať tóny a farby, stále je potrebný kvalifikovaný umelec, ako je tomu v prípade 
% dodatočného spracovania kinematografického filmu.

% Počítačové grafické techniky schopné vykresliť scény s vysokým kontrastom posunuli zameranie sa z farby na jas ako 
% hlavný obmedzujúci faktor zobrazovacích zariadení. Bolo vyvinutých niekoľko operátorov mapovania tónov na mapovanie 
% obrazov s vysokým dynamickým rozsahom na štandardné zobrazenia. Nedávno sa táto práca odklonila od využívania 
% jasu na rozšírenie kontrastu obrazu a smerom k iným metódam, ako je napríklad reprodukcia obrázkov podporovaná interakciou
% používateľa. V súčasnej dobe sa reprodukcia obrazu posunula smerom k riešeniam riadeným displejmi, pretože displeje 
% teraz disponujú pokročilými algoritmami spracovania obrazu, ktoré pomáhajú prispôsobiť vykresľovanie obrazu pre 
% podmienky zobrazenia, šetriť energiu, farebný a dynamický rozsah.

% \begin{figure}[t!]
%   \centering
%   \begin{subfigure}{0.3\textwidth}
%       \includegraphics[width=\textwidth]{figures/theory/tmo/tmo-bilateral}
%       \caption{Bilaterálny filter}
%   \end{subfigure}
%   ~
%   \begin{subfigure}{0.3\textwidth}
%       \includegraphics[width=\textwidth]{figures/theory/tmo/tmo-ashikhmin}
%       \caption{Ashikhmin}
%   \end{subfigure}
%   ~
%   \begin{subfigure}{0.3\textwidth}
%       \includegraphics[width=\textwidth]{figures/theory/tmo/tmo-drago}
%       \caption{Drago}
%   \end{subfigure}
%   ~
%   \begin{subfigure}{0.3\textwidth}
%       \includegraphics[width=\textwidth]{figures/theory/tmo/tmo-reinhard}
%       \caption{Reinhard}
%   \end{subfigure}
%   ~
%   \begin{subfigure}{0.3\textwidth}
%       \includegraphics[width=\textwidth]{figures/theory/tmo/tmo-pattanaik}
%       \caption{Pattanaik}
%   \end{subfigure}
%   ~
%   \begin{subfigure}{0.3\textwidth}
%       \includegraphics[width=\textwidth]{figures/theory/tmo/tmo-ward}
%       \caption{Ward}
%   \end{subfigure}
%   \caption{Rozličné operátory mapovania tónov použité na jednu scénu (zdroj: \cite{AHDR})}
%   \label{fig:tmo_examples}
% \end{figure}