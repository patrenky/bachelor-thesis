Okolie, ktoré vnímame, má vysoký dynamický rozsah svetla a farieb. Tmavé miesta bez osvetlenia neobsahujú takmer žiaden jas a naopak
scéna zameraná na zdroj svetla obsahuje priveľmi veľa jasu. Ľudské oko je schopné prispôsobiť sa takýmto zmenám a pozorovať detaily
aj na scéne s rozmanitým rozsahom jasu.

Väčšina digitálnych fotoaparátov a zobrazovacích zariadení však zachytávajú a zobrazujú farebné obrázky ako matice 24-bitových pixelov,
v ktorých je obsiahnutých 8-bitov v~rozsahu 0 až 255 pre každý farebný kanál. Preto sa digitálne fotoaparáty pokúšajú odhadnúť osvetlenie
a automaticky nastaviť čas expozície tak, aby mal najdôležitejší aspekt scény čo najlepší dynamický rozsah a jas miest, ktoré sú príliš
tmavé, alebo naopak príliš svetlé, je orezaný na hodnoty 0 a 255. Tento problém rieši HDR fotografia, avšak nie veľa bežných užívateľov si
je vedomých, čo to vlastne HDR fotografia znamená a ako sa s ňou pracuje.

HDR fotografia umožňuje zachytiť veľkú časť rozsahu jasu reálneho sveta a následnú prácu s týmito dátami. Existuje viacero mobilných aplikácií,
ktoré ponúkajú vytvorenie a~spracovanie HDR fotografie. Veľa verejne dostupných aplikácií však používa iba filter aplikovaný na jednu
fotografiu, ktorý zvýši kontrast farieb a detaily a tým sa snaží opticky vytvoriť efekt HDR. Na druhej strane sú aplikácie, ktoré
vytvárajú HDR fotografiu skladaním série snímok s rôznymi nastaveniami času expozície. Tieto aplikácie však poväčšine užívateľovi neposkytujú
dostatočne záživné užívateľské rozhranie, majú pre užívateľa veľmi obmedzené možnosti, alebo sa s nimi ťažko a neintuitívne pracuje.

Zameraním tejto práce je vytvoriť aplikáciu, ktorá by riešila nielen problémy generovania a spracovania HDR obsahu, ale zamerala sa aj na
interakciu s užívateľom a poskytla mu viac možností ako bežná aplikácia. Každá scéna je niečim výnimočná a jednoduché východzie nastavenia
hodnôt parametrov nedosiahnú vždy uspokojivé výsledky.
