Výber technológií pre implementáciu časovo a priestorovo náročných algoritmov je dôležitá a často 
náročná časť práce. Vzhľadom k tomu, že natívna aplikácia fotoaparátu pre iOS podporuje vytváranie
HDR fotografie, táto práca bude zameraná na platformu Android. Použité vývojové prostredie je
Android Studio, ktoré poskytuje širokú škálu nástrojov pre editovanie kódu a zároveň ponúka tvorbu 
viziálneho návrhu grafického užívateľského rozhrania. Zvoleným programovacím jazykom je objektovo 
orientovaná Java. Pre riešenie náročných matematických a grafických operácií je použitá knižnica
OpenCV. Knižnica je implementovaná v jazyku C++, čo znamená, že metódy sú v operačnom systéme Android
vykonávané natívne a tým nám prináša výhodu v rýchlosti spracovania.

Aplikácia implementuje štandardný postup vytvárania HDR fotografie na základe spájania série snímok
(obr. \ref{fig:app_pipeline}). Užívateľské rozhranie aplikácie poskytuje užívateľovi možnosti
interaktivity pri vytváraní výslednej fotografie. Pre zarovnanie snímok je použitá metóda Median
Threshold Bitmap od Grega Warda (viac v podkapitole \ref{sec:Practice-Alignment}). 
Využitím krivky odozvy zariadenia, ktorú dosiahneme metódou od Paula E. Debeveca a Jitendru Malika,
popísanou v podkapitole \ref{sec:Theory-Generating} a váhovej funkcie, vygenerujeme HDR obsah,
ktorý je sám o~sebe nezobraziteľný. Vygenerovaný HDR obsah možno zobraziť pomocou operátorov mapovania
tónov. Aplikácia implementuje celkovo štyri operátory mapovania tónov: globálny operátor Erika Reinharda,
operátor Frederika Draga a lokálne operátory od Duranda a~Mantiuka (podkapitola \ref{sec:Theory-Operators}).

Zdrojový kód aplikácie je členený podľa architektonického vzoru model-view-controller, ktorý rozdeľuje
aplikáciu do troch hlavných častí:
\begin{description}
  \item [Dátový model] uchováva dáta naprieč aplikáciou a operácie nad týmito dátami.
\end{description}
\setlength{\DTbaselineskip}{2em}
\dirtree{%
.1 model.
.2 CameraCRF \DTcomment{Krivka odozvy zariadenia}.
.2 Exposures \DTcomment{Vybrané expozičné časy pre aktuálnu scénu}.
.2 ImageHDR \DTcomment{HDR obsah}.
.2 ImageLDR \DTcomment{Vytvorená fotografia}.
.2 TmoParams \DTcomment{Východzie hodnoty a prevod relatívnych hodnôt na absolútne}.
}

\begin{description}
  \item [Užívateľské rozhranie] umožňuje prezentáciu dát. V našom prípade \texttt{view} obsahuje
  triedy pracujúce s Android Fragmentmi (viac v podkapitole \ref{sec:Practice-UI}) a dialógovými oknami.
\end{description}
\setlength{\DTbaselineskip}{2em}
\dirtree{%
.1 view.
.2 CameraFragment.
.2 EditDragoFragment.
.2 EditDurandFragment.
.2 EditMantiukFragment.
.2 EditReinhardFragment.
.2 FilesFragment.
.2 HomeFragment.
.2 SaveDialog.
.2 SettingsDialog.
.2 TmoFragment.
}

\begin{description}
  \item [Riadiaca logika aplikácie] riadi tok udalostí v programe a modifikuje dáta v dátovom modeli.
\end{description}
\setlength{\DTbaselineskip}{2em}
\dirtree{%
.1 controller.
.2 camera.
.3 AlignImages \DTcomment{Zarovnanie série fotografií}.
.2 hdr.
.3 CRFRecover \DTcomment{Získanie krivky odozvy zariadenia}.
.3 HDRController \DTcomment{Kontrolér postupu vytvárania HDR obsahu}.
.3 HDRMerge \DTcomment{Generovanie HDR obsahu}.
.3 SamplesSelector \DTcomment{Výber vzorky pixelov}.
.2 tmo \DTcomment{Metódy operátov mapovania tónov}.
.3 TMODrago.
.3 TMODurand.
.3 TMOMantiuk.
.3 TMOReinhard.
.2 Convertor \DTcomment{Prevody medzi dátovými typmi}.
.2 Storages \DTcomment{Správa úložísk zariadenia}.
}

\begin{figure}[t]
  \centering
  \includegraphics[width=1\textwidth]{figures/ui/pipeline}
  \caption{Postup vytvárania HDR fotografie pomocou aplikácie}
  \label{fig:app_pipeline}
\end{figure}